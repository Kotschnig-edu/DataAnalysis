\documentclass[11pt]{article}
\usepackage[utf8]{inputenc}
\usepackage{geometry}
\usepackage{graphicx}
\usepackage{booktabs}
\usepackage{hyperref}
\usepackage{amsmath}
\usepackage{enumitem}
\usepackage{xcolor}

\geometry{a4paper, margin=1in}

\title{Data Analysis Final Assignment Report}
\author{% 
  Team: Analog Avengers\\ 
  Eingang Fabian
  \And
  Kotschnig Thomas
  \And 
  Krenn Matthias 
  }

\date{}

\begin{document}
\maketitle

\begin{center}
\textit{Note: This template provides a suggested structure aligned with the current task categories. You may adjust headings if needed and please ensure all required components are covered.}
\end{center}

\section{Contributions}
\textit{Clearly state each team member's specific contributions. Be concrete.}
\begin{itemize}[leftmargin=*]
    \item Eingang Fabian:
    \begin{itemize}
        \item Dataset selection and acquisition
        \item Data quality analysis and preprocessing pipeline
    \end{itemize}
    \item Kotschnig Thomas:
    \begin{itemize}
        \item Visualizations and EDA
        \item Probability analysis tasks
    \end{itemize}
    \item Krenn Matthias:
    \begin{itemize}
        \item Regression modeling and interpretation
        \item Report writing and figure polishing
    \end{itemize}
\end{itemize}

\section{Dataset Description}
\begin{itemize}[leftmargin=*]
    \item "Bike sales in Europe" from https://www.kaggle.com/datasets/sadiqshah/bike-sales-in-europe
    \item It has more than 100k entries of sales data from different countries. Streching from 2011 to 2016, with a daily sampling frequency.
    \item Key variables analyzed: unit price, unit cost, date, order quantity, customer age, product type
    \item Shape: 113036 rows x 18 columns
    \item No missing data, however, the entry of some dates is missing completely. 
    This resolves in no missing data, but inconsistent time series. There is only one bigger gap, therefore we decided 
    \item Any known limitations or caveats:
\end{itemize}

\section{Task 1. Data Preprocessing and Basic Analysis}

\subsection{Basic statistical analysis using pandas}
\begin{itemize}[leftmargin=*]
    \item Descriptive stats (mean, std, min, max, quantiles) for key variables:
    \item Grouped summaries where relevant (by day, device, category, test run):
\end{itemize}

\subsection{Original data quality analysis including visualization}
\begin{itemize}[leftmargin=*]
    \item Missingness patterns (counts, heatmap, timeline gaps):
    \item Outliers and suspicious values (plots and rule used):
    \item Consistency checks (timestamps order, duplicates, impossible values):
\end{itemize}

\subsection{Data preprocessing}
\begin{itemize}[leftmargin=*]
    \item Cleaning steps performed:
    \item Missing-value treatment (drop, impute, interpolate, forward fill, etc.):
    \item Outlier handling (range, threshold, IQR, percentile, justify choice):
    \item Feature engineering (e.g., scaling/normalization, log):
    \item Final dataset shape after preprocessing:
\end{itemize}

\subsection{Preprocessed vs original data visual analysis}
\begin{itemize}[leftmargin=*]
    \item Before vs after comparison plots (at least 2 to 3 key variables):
    \item What improved and what trade-offs exist:
\end{itemize}

\section{Task 2. Visualization and Exploratory Analysis}

\subsection{Time series visualizations}
\begin{itemize}[leftmargin=*]
    \item Plot of main variable(s) over time:
    \item Annotations for notable events or pattern shifts (if applicable):
\end{itemize}

\subsection{Distribution analysis with histograms}
\begin{itemize}[leftmargin=*]
    \item Histograms for key numeric variables:
    \item Notes on skewness, heavy tails, multi-modality:
\end{itemize}

\subsection{Correlation analysis and heatmaps}
\begin{itemize}[leftmargin=*]
    \item Correlation type used (Pearson or Spearman) and why:
    \item Heatmap and top correlated pairs with short interpretation:
\end{itemize}

\subsection{Daily pattern analysis}
\begin{itemize}[leftmargin=*]
    \item Aggregation method (hourly means, day-of-week, rolling averages):
    \item Plots showing daily cycles or weekday-weekend differences:
    \item What patterns are stable vs noisy:
\end{itemize}

\subsection{Summary of observed patterns, similar to True/False questions}
\textit{Write short, testable statements and answer them based on evidence. Example format below.}
\begin{itemize}[leftmargin=*]
    \item Statement 1 (True or False): \textbf{...}. Evidence: ...
    \item Statement 2 (True or False): \textbf{...}. Evidence: ...
    \item Statement 3 (True or False): \textbf{...}. Evidence: ...
\end{itemize}

\section{Task 3. Probability Analysis}

\subsection{Threshold-based probability estimation}
\begin{itemize}[leftmargin=*]
    \item Define threshold(s) and justify choice:
    \item Estimate probabilities of exceeding thresholds:
    \item Visual support (e.g., empirical CDF, bar plot, timeline highlights):
\end{itemize}

\subsection{Cross tabulation analysis}
\begin{itemize}[leftmargin=*]
    \item Define two categorical variables (or binned numeric variables):
    \item Present contingency table and interpret key cells:
\end{itemize}

\subsection{Conditional probability analysis}
\begin{itemize}[leftmargin=*]
    \item Define events $A$ and $B$:
    \item Compute and interpret $P(A)$, $P(B)$, $P(A \mid B)$, $P(B \mid A)$:
    \item Include at least one meaningful comparison and conclusion:
\end{itemize}

\subsection{Summary of observations from each probability task}
\begin{itemize}[leftmargin=*]
    \item Key takeaway from threshold probability:
    \item Key takeaway from crosstab:
    \item Key takeaway from conditional probability:
\end{itemize}

\section{Task 4. Statistical Theory Applications}

\subsection{Law of Large Numbers (LLN) demonstration}
\begin{itemize}[leftmargin=*]
    \item Variable chosen and why it makes sense:
    \item Experiment: show sample mean as $n$ increases:
    \item Plot and short interpretation:
\end{itemize}

\subsection{Central Limit Theorem (CLT) application}
\begin{itemize}[leftmargin=*]
    \item Sampling procedure (sample size, number of trials, with or without replacement):
    \item Show distribution of sample means for increasing $n$:
    \item Plot(s): histogram(s) of sample means and comparison to normal shape:
\end{itemize}

\subsection{Result interpretation}
\begin{itemize}[leftmargin=*]
    \item What LLN showed in your data context:
    \item What CLT showed, and any deviations and why:
\end{itemize}

\section{Task 5. Regression Analysis}

\subsection{Linear or Polynomial model selection}
\begin{itemize}[leftmargin=*]
    \item Define target $y$ and predictors $X$:
    \item Motivation for linear vs polynomial:
    \item Any train-test split rationale (time-aware split if relevant):
\end{itemize}

\subsection{Model fitting and validation}
\begin{itemize}[leftmargin=*]
    \item Fit procedure and preprocessing (scaling, feature selection):
    \item Validation method (holdout, time-series split, etc.):
    \item Metrics reported (RMSE, MAE, $R^2$) and why:
    \item Residual analysis (at least one plot recommended):
\end{itemize}

\subsection{Result interpretation and analysis}
\begin{itemize}[leftmargin=*]
    \item Main effects and practical meaning:
    \item Failure cases or where model performs poorly:
\end{itemize}

\section{Bonus Tasks}
\begin{itemize}[leftmargin=*]
    \item New dataset bonus (10): state why dataset is new and provide link:
    \item Q-Q plot with explanation (5):
    \begin{itemize}
        \item Either for CLT sample means, or regression residuals:
        \item Interpretation of deviations from normality:
    \end{itemize}
    \item Interactive visualizations (up to 10): describe tool used and what interactivity adds:
    \item Cross-validation in regression (5): method used and how results compare to holdout:
    \item Additional exploration (up to 20): clearly state extra tasks and value gained:
\end{itemize}

\section{Key Findings and Conclusions}
\begin{itemize}[leftmargin=*]
    \item Main findings from preprocessing and EDA:
    \item Main findings from probability tasks:
    \item Main findings from LLN and CLT:
    \item Main findings from regression:
    \item Limitations:
    \item What you would do next if you had more time:
\end{itemize}

\section{Reproducibility Notes}
\begin{itemize}[leftmargin=*]
    \item Exact dataset source link and version or download date:
    \item Key libraries used and versions (optional but recommended):
    \item How to run the notebook end-to-end:
\end{itemize}

% Optional
% \section{References}
% \begin{enumerate}
%     \item ...
% \end{enumerate}

\end{document}
